%%%%%%%%%%%%%%%%%%%%%%%%%%%%%%%%%%%%%%%%%%%%%%%%%%%%%%%%%%%%%%
%                                                            %
%                    Read constant commands                  %
%                                                            %
%%%%%%%%%%%%%%%%%%%%%%%%%%%%%%%%%%%%%%%%%%%%%%%%%%%%%%%%%%%%%%

% Logo 

\newcommand{\cnsFiveTwelve}{resources/images/logo/data-mining}

\newcommand{\figone}{resources/images/Figure_1}
\newcommand{\figtwo}{resources/images/Figure_2}
\input{resources/constants/images path/IMAGE_NAMES.tex}
\input{resources/constants/images path/IMAGE_REFS.tex}
% User manual configuration

\newcommand{\userManualSimpleTextSize}{13pt}
\newcommand{\userManualSimpleTextStyle}{Times New Roman}

% This command defines custom font sizes for section, subsection, and subsubsection headings in LaTeX, and sets them accordingly using the titlesec package.
% Usage: Simply call the \myheadingstyles command in the preamble of your LaTeX document to define and set the custom font sizes.
% Arguments: None.

\newcommand{\myheadingstyles}{
	\newcommand{\sectionfont}{\fontsize{20}{24}\selectfont}
	\newcommand{\subsectionfont}{\fontsize{18}{22}\selectfont}
	\newcommand{\subsubsectionfont}{\fontsize{16}{20}\selectfont}
	
	\titleformat{\section}
	{\sectionfont\bfseries}{\thesection}{1em}{}
	\titleformat{\subsection}
	{\subsectionfont\bfseries}{\thesubsection}{1em}{}
	\titleformat{\subsubsection}
	{\subsubsectionfont\bfseries}{\thesubsubsection}{1em}{}
}

\newcommand{\myTOC}{
	% Set font size and style for the table of contents
	\renewcommand{\contentsname}{\fontsize{14}{16}\selectfont\textbf{\fontfamily{ptm}\selectfont Table of Contents}}
	\Large
	% Print the table of contents
	\tableofcontents
	
	\newpage
	

}

%%%%%%%%%%%%%%%%%%%%%%%%%%%%%%%%%%%%%%%%%%%%%%%%%%%%%%%%%%%%%%
%                                                            %
%                       Custom Commands                      %
%                                                            %
%%%%%%%%%%%%%%%%%%%%%%%%%%%%%%%%%%%%%%%%%%%%%%%%%%%%%%%%%%%%%%

% Creates a new command for adding an animated GIF to the document.
% Usage: \sectionCenteredgif[optional scale]{GIF file name}{caption text}{label}
% - The optional scale argument (default 0.9) can be used to adjust the width of the GIF.
% - The GIF file should be in the same directory as the LaTeX document.
% - The caption text should describe the contents of the GIF.
% - The label should be a unique identifier for the GIF, used for referencing it later in the text.

\newcommand{\sectionCenteredgGif}[4][0.9]{
	\begin{figure}[H]
		\centering
		\animategraphics[width=#1\linewidth,autoplay,loop]{12}{#2}{0}{49}
		\caption{#3}
		\label{fig:#4}
	\end{figure}
}


% Usage: \imageCaptionTable{image path}{image caption}{image label}{text}
% - The first argument should be the path to the image file in any format.
% - The second argument should be the caption of the image to be displayed.
% - The third argument should be the label of the image.
% - The fourth argument should be the text to be displayed on the right side of the image.
% - The resulting table will have the image displayed on the left (with the caption and label) and the text on the right.

\newcommand{\imageCaptionTable}[4]{%
	\begin{tabular}{l l}
		\includegraphics[width=0.5\textwidth, caption={#2}, label={#3}]{#1} & \centering
		\begin{varwidth}[t]{0.8\textwidth}
			\configuratedText{#4}
		\end{varwidth}
	\end{tabular}
}

% Inserts a small image in the middle of a text
% Usage: \smallimage[optional scale]{image file name}
% - The optional scale argument (default 0.5) can be used to adjust the size of the image.
% - The image file should be in the same directory as the LaTeX document.

\newcommand{\smallimage}[2][0.03]{
	\includegraphics[width=#1\linewidth]{#2}
}


% Creates a new figure with an image centered on the page, and adds a caption and label for reference.
% Usage: \sectionCenteredfigure[optional scale]{image file name}{caption text}{label}
% - The optional scale argument (default 0.9) can be used to adjust the width of the image.
% - The image file should be in the same directory as the LaTeX document.
% - The caption text should describe the contents of the image.
% - The label should be a unique identifier for the figure, used for referencing it later in the text.

\newcommand{\sectionCenteredfigure}[4][0.9]{
	\begin{figure}[H]
		\centering
		\fbox{\includegraphics[width=#1\linewidth]{#2}}
		\caption{#3}
		\label{fig:#4}
	\end{figure}
}


% Creates a new block of justified text with a specified font and font size.
% Usage: \generalText{font family}{font size}{text}
% - The font family argument specifies the font to be used (e.g., Times New Roman, Arial, etc.).
% - The font size argument specifies the size of the font (e.g., 12pt, 14pt, etc.).
% - The text argument should contain the text to be justified.
% - The resulting block of text will be fully justified (i.e., aligned with both the left and right margins).

\newcommand{\customText}[3]{%
	\par\begingroup
	\setlength{\parindent}{0pt}%
	\linespread{1.3}%
	\fontsize{#2}{#2}%
	\fontfamily{#1}\selectfont #3%
	\par\endgroup%
}

% Creates a new block of justified text with the font size and font style of configuration file.
% Usage: \generalText{text}
% - The font family argument is specfied by \userManualSimpleTextStyle 
% - The font size argument is specified by \userManualSimpleTextSize
% - The text argument should contain the text to be justified.
% - The resulting block of text will be fully justified (i.e., aligned with both the left and right margins).

\newcommand{\configuratedText}[1]{%
	\par\begingroup
	\setlength{\parindent}{0pt}%
	\linespread{1.3}%
	\selectfont #1%
	\par\endgroup%
}

% Defines a new command for referencing figures.
% Usage: \figref{label}
% - The argument should be the label of the figure to be referenced.
% - The resulting output will be in the format "(Figure <number>)", where <number> is the number of the referenced figure.
% - The label should be defined using \label{fig:<label>} command in the figure environment.
% - The \hyperref command creates a hyperlink to the referenced figure.
% - The \ref* command produces only the number of the referenced figure, without the preceding "Figure" text.

\newcommand{\figref}[1]{(\hyperref[fig:#1]{Figure \ref*{fig:#1}})}

% Creates a new table with an icon and its name.
% Usage: \customTable{icon path}{icon name}
% - The first argument should be the path to the icon file in PNG format.
% - The second argument should be the name of the icon to be displayed.
% - The resulting table will have the icon displayed on the left and its name on the right.

\newcommand{\iconNameTable}[2]{%
	\begin{tabular}{l l}
		\includegraphics[width=0.03\textwidth]{#1} & \centering 
		\begin{varwidth}[t]{0.8\textwidth}
			\configuratedText{#2}
		\end{varwidth}
	\end{tabular}
}

% Creates a new table with an icon and its name.
% Usage: \customTable{icon path}{icon name}
% - The first argument should be a text.
% - The second argument should be a text.
% - The resulting table will have the text displayed on the left and a text on the right.

\newcommand{\textTextTable}[3][2cm]{%
	\begin{tabular}{p{#1} p{\dimexpr0.90\textwidth-#1}}
		\configuratedText{#2}
		&
		\begin{varwidth}[t]{\linewidth}
			\configuratedText{#3}
		\end{varwidth}
	\end{tabular}
}

% Creates a new table with an icon, its name, and its description.
% Usage: \iconNameDescriptTable{icon path}{icon name}{icon description}
% - The first argument should be the path to the icon file in PNG format.
% - The second argument should be the name of the icon to be displayed.
% - The third argument should be a description of the icon.
% - The resulting table will have the icon displayed on the left, its name in the middle, and its description on the right.
% - The table has three columns with widths of 0.1, 0.3, and 0.5 times the text width, respectively.
% - The second and third columns are aligned to the left.

\newcommand{\iconNameDescriptTable}[3]{%
	\begin{tabular}{p{0.05\textwidth} p{0.2\textwidth} m{0.6\textwidth}}
		\includegraphics[width=0.03\textwidth]{#1} & \raggedright \configuratedText{#2} & \justify \configuratedText{#3} \
	\end{tabular}
}

% Creates a new table with an text, its name, and its description.
% Usage: \textDescriptTable{text}{text name}{text description}
% - The first argument should be the text to be displayed on the left.
% - The second argument should be the name of the text.
% - The third argument should be a description of the text.
% - The resulting table will have the text displayed on the left, its name in the middle, and its description on the right.
% - The table has three columns with widths of 0.1, 0.3, and 0.5 times the text width, respectively.
% - The second and third columns are aligned to the left.

\newcommand{\textDescriptTable}[3]{%
	\begin{tabular}{m{0.1\textwidth} m{0.1\textwidth} m{0.5\textwidth}}
		\raggedright \configuratedText{#1} & \raggedright \configuratedText{#2} & \raggedright \configuratedText{#3} \
	\end{tabular}
}

% Creates a new block of two columns with an image on the left and justified text on the right.
% Usage: \twocolumns{optional scale}{image file name}{caption text}{font family}{font size}{text}

% - 1: The optional scale argument (default 0.9) can be used to adjust the width of the image.
% - 2: The image file should be in the same directory as the LaTeX document.
% - 3: The caption text should describe the contents of the image.
% - 4: The font family argument specifies the font to be used (e.g., Times New Roman, Arial, etc.).
% - 5: The font size argument specifies the size of the font (e.g., 12pt, 14pt, etc.).
% - 6: The text argument should contain the text to be justified.

\newcommand{\twoColumns}[6]{
	\begin{minipage}[t]{0.50\textwidth}
		\customText{#4}{#5}{#6}
	\end{minipage}\hfill
	\begin{minipage}[r]{0.45\textwidth}

	\end{minipage}
}

\documentclass[12]{article}
\usepackage[utf8]{inputenc}
\usepackage{graphicx}
\usepackage{geometry}
\usepackage{tocloft}
\usepackage{amsmath}
\usepackage{booktabs}
\usepackage{fancyhdr}
\usepackage{hyperref}
\usepackage{xcolor}
\usepackage{soul}
\usepackage{times}
\usepackage{listings}
\usepackage{url}
\usepackage{wrapfig}
\usepackage{array}
\usepackage{varwidth}
\usepackage{float}
\usepackage{titlesec}
\usepackage{ragged2e}
\usepackage{todonotes}
\usepackage{tocloft}
\usepackage{changepage}
\usepackage{animate}
\usepackage{forest}
\usepackage{adjustbox}

\usepackage{xcolor}
\usepackage{tcolorbox}
\usepackage{hyperref}

\lstset{
	breaklines=true, 
	basicstyle=\ttfamily,
	columns=fullflexible
}

\cftsetindents{section}{1.5em}{5.0em}
\cftsetindents{subsection}{2em}{5.0em}
\cftsetindents{subsubsection}{3em}{5.0em}

% Use the new command to set the font sizes and titleformat settings
\myheadingstyles

% Customize hyperlinks in the document
\hypersetup{
	colorlinks=true, % enable colored hyperlinks
	linkcolor=blue, % set the color of internal links to black
	filecolor=magenta, % set the color of links to local files to magenta
	urlcolor=blue, % set the color of links to URLs to blue
	bookmarks=true, % enable the creation of bookmarks in the PDF file
}

\geometry{
	a4paper, % set paper size to A4
	left=2cm, % set left margin to 2cm
	right=2cm, % set right margin to 2cm
	top=2.5cm, % set top margin to 2.5cm
	bottom=2.5cm % set bottom margin to 2.5cm
}

\lstset{
	basicstyle=\ttfamily\normalsize, % Monospace font
	breaklines=true,            % Enable line breaking
	columns=fullflexible,       % Avoids text spacing issues
	keepspaces=true
}

\tcbset{
	colback=gray!10, % Background color
	colframe=gray!50, % Border color
	boxrule=0.5mm, % Border thickness
	sharp corners,
	left=5pt, % Left padding
	right=5pt % Right padding
}

%%%%%%%%%%%%%%%%%%%%%%%%%%%%%%%%%%%%%%%%%%%%%%%%%%%%%%%%%%%%%%
%                                                            %
%                    Document Starts Here                    %
%                                                            %
%%%%%%%%%%%%%%%%%%%%%%%%%%%%%%%%%%%%%%%%%%%%%%%%%%%%%%%%%%%%%%

\begin{document}
	
	\begin{center}
		\begin{figure}[h]
			\centering
			\includegraphics[width=0.4\linewidth]{\cnsFiveTwelve}
		\end{figure}
		\vspace{1cm} 
		{\fontsize{28}{34}\selectfont \textbf{Data Mining}}
	\end{center}

	\vspace{0.5cm} 
	
	\begin{center}
	{\fontsize{22}{28}\selectfont \textbf{HW 4}}
	\end{center}

	\vspace{0.5cm} 

	\begin{center}
	{\fontsize{20}{25}\selectfont Ashkan Ansarifard}
	\end{center}
	\vspace{0.5cm} 

	\begin{center}
		{\fontsize{20}{25}\selectfont 1970082}
	\end{center}

	\vspace{1cm} 

	\textcolor{blue!60!black}{\rule{\linewidth}{2pt}}
	
	\vspace{8cm} 
	
	\begin{center}
	\textbf{A.Y. 2024/25}
	\end{center}

	\thispagestyle{empty}
	
	\newpage
	
	\myTOC

	\newpage
	\section{Introduction}\label{sec:intro}
	\subsection{Project Structure}\label{sec:proj-strut}
	\configuratedText{In this section, we show how the project is structured. In the following sections, we refect to this structure and explain what each module does.
		
	For now, you can find the overall project structure as following:\\
	
	}
	\begin{lstlisting}
		project_root/
		+-- main.py
		|
		+-- data/
		|   +-- dataset_analyzer.py
		|   +-- dataset_loader.py
		|   +-- prepare_link_prediction_data.py
		|
		+-- embeddings/
		|   +-- analyze_embeddings.py
		|   +-- visualize_embeddings.py
		|
		+-- evaluation/
		|   +-- evaluator.py
		|   +-- cross_dataset_evaluator.py
		|   +-- link_prediction_evaluator.py
		|
		+-- models/
		|   +-- gnn_model.py
		|   +-- generalized_gnn.py
		|   +-- gnn_explainer.py
		|   +-- node2vec_model.py
		|   +-- link_prediction_gnn.py
		|
		+-- saved_models/
		|   +-- generalized_<timestamp>/
		|       +-- model.pth
		|       +-- metadata.json
		|
		+-- training/
		|   +-- trainer.py
		|   +-- trainer_link_prediction.py
		|
		+-- utils/
		|   +-- config.py
		|   +-- dimensionality_handler.py
		|   +-- device.py
		|   +-- find_best_models.py
		|   +-- load_generalized_gnn.py
		|   +-- model_saver.py
		|
		+-- README.md
		+-- requirements.txt
		+-- setup.bat
		+-- setup.sh
	\end{lstlisting}
	
	\subsection{Environment Setup}\label{sec:env-setup}
	
	To execute this project, follow these steps:
	
	\begin{enumerate}
		\item Ensure that Python 11.3 is installed on your system. You can download it from the \href{https://www.python.org/downloads/}{official Python website}.
		
		\item Run the provided setup script to create and configure the virtual environment:
		\begin{itemize}
			\item On Windows, run:
			\begin{tcolorbox}
				\texttt{setup.bat}
			\end{tcolorbox}
			\item On macOS/Linux, run:
			\begin{tcolorbox}
				\texttt{bash setup.sh}
			\end{tcolorbox}
		\end{itemize}
		
		\item After the setup script completes, activate the virtual environment:
		\begin{itemize}
			\item On Windows:
			\begin{tcolorbox}
				\texttt{<venv\_name>\textbackslash Scripts\textbackslash activate}
			\end{tcolorbox}
			\item On macOS/Linux:
			\begin{tcolorbox}
				\texttt{source <venv\_name>/bin/activate}
			\end{tcolorbox}
		\end{itemize}
		
		\item Run the project using:
		\begin{tcolorbox}
			\texttt{python main.py}
		\end{tcolorbox}
		If executed without providing the required arguments, the following message will be displayed:
		\begin{tcolorbox}[colback=gray!10, colframe=gray!50, sharp corners, boxrule=0.5mm]
			\texttt{usage: main.py [-h] \{single, \\
				experiment, \\
				find\_best, \\
				link\_prediction, \\
				analyze\_embeddings, \\
				node2vec,explain\} ...\\
				main.py: error: the following arguments are required: command}
		\end{tcolorbox}
		
	\end{enumerate}
	
	After completing these steps, your environment will be properly configured for executing the project.
	
	\newpage
	\section{Dataset Analyzer}\label{sec:dataset-metrics-analyzer}
	
	The \texttt{DatasetMetricsAnalyzer} calculates various graph metrics to analyze the structure and properties of datasets. 
	
	\subsubsection*{Metrics and Their Interpretations}
	\begin{itemize}
		\item \textbf{Degree Distribution:} \\
		Describes the distribution of node degrees in the graph. A similar degree distribution across datasets may indicate similar graph connectivity patterns, which could improve model transferability.
		
		\item \textbf{Average Clustering Coefficient:} \\
		Measures the likelihood of nodes forming local triangles. Higher values indicate dense local clusters. Comparing this across datasets helps understand differences in local graph density, affecting message-passing algorithms.
		
		\item \textbf{Graph Density:} \\
		The ratio of edges to possible edges in the graph. Densely connected graphs may allow models to propagate information more effectively, while sparse graphs may pose challenges for learning.
		
		\item \textbf{Connected Components:} \\
		Counts the number of isolated subgraphs in the dataset. Fewer components indicate better connectivity, which often correlates with improved model performance for tasks like node classification.
		
		\item \textbf{Community Detection (Louvain):} \\
		Identifies clusters or communities in the graph and measures modularity, which evaluates the strength of these communities. Higher modularity indicates well-defined clusters, which can aid in tasks like community detection or semi-supervised learning.
		
		\item \textbf{Diameter and Average Path Length:} \\
		The diameter is the longest shortest path between any two nodes, while the average path length indicates the typical distance between nodes. Smaller values suggest more compact graphs, potentially leading to faster information propagation.
		
		\item \textbf{Node Centrality Metrics:} \\
		Measures like degree centrality, betweenness centrality, closeness centrality, eigenvector centrality, and PageRank identify important nodes in the graph. Comparing centrality metrics across datasets highlights differences in node influence and structural roles, affecting task-specific outcomes.
		
		\item \textbf{Assortativity:} \\
		Quantifies the tendency of nodes to connect with similar-degree nodes. Datasets with high assortativity may require models that leverage this homophily for better performance.
		
		\item \textbf{Transitivity:} \\
		Measures the ratio of triangles to possible triangles in the graph. High transitivity indicates a tightly interconnected graph, which can enhance tasks like link prediction.
		
		\item \textbf{Radius and Core Numbers:} \\
		The radius measures the minimum eccentricity of any node, while core numbers represent the maximum subgraph where all nodes have at least a certain degree. These metrics reveal the robustness and central structure of the graph.
		
		\item \textbf{Edge Betweenness:} \\
		Measures the importance of edges in connecting different parts of the graph. Datasets with high edge betweenness may be more sensitive to edge removal, impacting model generalization.
	\end{itemize}
	
	\subsubsection*{Benefits of Comparing Metrics Across Datasets}
	Comparing these metrics provides insights into:
	\begin{itemize}
		\item \textbf{Structural Similarity:} \\
		Helps identify datasets with similar graph structures, improving model transferability.
		\item \textbf{Feature Distribution:} \\
		Certain metrics, like clustering coefficient or transitivity, reflect underlying feature distributions and alignment.
		\item \textbf{Task Feasibility:} \\
		Datasets with high connectivity, well-defined communities, or uniform centrality may favor tasks like node classification or link prediction.
		\item \textbf{Performance Bottlenecks:} \\
		Metrics such as high diameter or low density can highlight challenges in propagating information through the graph, identifying areas where models may struggle.
		\item \textbf{Dataset Diversity:} \\
		Understanding the variations between datasets ensures that models are tested under diverse conditions, improving their robustness and generalization.
	\end{itemize}
	

	
	\newpage
	\section{Node Classification on Cora Dataset}
	
	\subsection{Simple GNN Model}
	\configuratedText{
	For this task, my first attempt was to create a "simple" GNN model, which loads the datasets from `data/datasets` (the pipeline will download them if it cannot find them). Then it creates an instance of the `GNNModel` with fixed characteristics. 
	
	The architecture of the model consists of two graph convolution layers. Specifically:
	\begin{itemize}
		\item The first layer is a GCNConv layer that maps the input node features (\texttt{input\_dim}) to a hidden representation (\texttt{hidden\_dim}).
		\item A ReLU activation is applied to introduce non-linearity, followed by a dropout layer to prevent overfitting.
		\item The second layer is another GCNConv layer that maps the hidden representation to the output space (\texttt{output\_dim}), corresponding to the number of target classes.
		\item A log-softmax activation function is applied to the final layer's output to produce normalized probabilities over the node classes.
	\end{itemize}
	
	The model takes as input:
	\begin{itemize}
		\item \texttt{x}: The feature matrix of the nodes.
		\item \texttt{edge\_index}: The edge list that represents the graph structure.
	\end{itemize}
	}
	
	\subsection{Generalized GNN Model}
	\configuratedText{
	The next step, was to make a more "Generalized" model which can be initiated with different variabless.	
	The Generalized GNN model is a flexible and modular implementation that supports multiple GNN variants, including GCN, GraphSAGE, and GAT. The architecture can be customized in terms of the number of layers, the use of residual connections, layer normalization, and dropout. Below, the key design features and characteristics of the model are outlined:
	
	\begin{itemize}
		\item \textbf{GNN Variants}: The model supports three types of graph convolution layers:
		\begin{itemize}
			\item \texttt{GCNConv} for Graph Convolutional Networks (GCN).
			\item \texttt{SAGEConv} for GraphSAGE.
			\item \texttt{GATConv} for Graph Attention Networks (GAT).
		\end{itemize}
		
		\item \textbf{Architecture}:
		\begin{itemize}
			\item The first layer maps the input node features (\texttt{input\_dim}) to a hidden representation (\texttt{hidden\_dim}).
			\item Intermediate layers (optional) maintain the hidden dimension and apply the selected convolution operator.
			\item The final layer maps the hidden representation to the output space (\texttt{output\_dim}), producing class logits for each node.
		\end{itemize}
		
		\item \textbf{Features and Enhancements}:
		\begin{itemize}
			\item \textbf{Residual Connections}: If enabled, adds a residual skip connection between the input and hidden layers, facilitating gradient flow and stabilizing training.
			\item \textbf{Layer Normalization}: Optionally applies layer normalization after each hidden layer to improve training dynamics.
			\item \textbf{Dropout}: Introduces dropout regularization after ReLU activations to prevent overfitting.
		\end{itemize}
		
		\item \textbf{Input Flexibility}:
		\begin{itemize}
			\item Supports either a full \texttt{data} object containing node features (\texttt{x}) and edge indices (\texttt{edge\_index}) or separate \texttt{x} and \texttt{edge\_index} tensors.
		\end{itemize}
		
		\item \textbf{Output}: The model outputs log-softmax probabilities across node classes, making it suitable for node classification tasks.
	\end{itemize}
	
	This generalized design allows for experimentation with various GNN architectures, making it adaptable for different datasets and graph-based learning tasks.
	}
	
	\subsection{Trainer Class}
	\configuratedText{
	The \texttt{GNNTrainer} class is designed to manage the training process for various types of GNN models, including both the simple GNN model and the Generalized GNN model. Its purpose is to streamline the process of loading data, initializing models,  optimizing parameters, and saving trained models along with metadata.
	
	\subsubsection{Key Features}
	
	\begin{itemize}
		\item \textbf{Dataset Handling}:
		\begin{itemize}
			\item Automatically loads the dataset (default: \texttt{Cora}) using the \texttt{DatasetLoader} utility.
		\end{itemize}
		
		\item \textbf{Model Initialization}:
		\begin{itemize}
			\item Supports two types of models:
			\begin{enumerate}
				\item \texttt{GNNModel}: A simpler two-layer GNN.
				\item \texttt{GeneralizedGNN}: A more customizable model supporting GCN, GraphSAGE, and GAT layers, with options for residual connections, layer normalization, and dropout.
			\end{enumerate}
			\item Model parameters such as the number of layers, hidden dimensions, and variant type can be configured during initialization.
		\end{itemize}
		
		\item \textbf{Training Process}:
		\begin{itemize}
			\item Utilizes the Adam optimizer with configurable learning rate and weight decay.
			\item Performs node classification by optimizing the negative log-likelihood loss function (\texttt{nll\_loss}).
			\item Displays training progress per epoch, including the loss value.
		\end{itemize}
		
		\item \textbf{Model and Metadata Saving}:
		\begin{itemize}
			\item Saves the trained model and its metadata (e.g., architecture details, dataset statistics, and training hyperparameters) for reproducibility and future use.
			\item Creates a dedicated directory to store models and metadata.
		\end{itemize}
	\end{itemize}
	
	\subsubsection{Workflow}
	
	The \texttt{GNNTrainer} class follows a systematic workflow:
	\begin{enumerate}
		\item Load the dataset.
		\item Initialize the model based on the specified type and configuration.
		\item Train the model for a fixed number of epochs, tracking the loss.
		\item Save the trained model and its metadata for evaluation or reuse.
	\end{enumerate}}

	\subsection{GNN Evaluator}
	\configuratedText{
	The \texttt{GNNEvaluator} class is designed to evaluate the performance of trained GNN models on node classification tasks. It loads a pre-trained model, applies it to a dataset, and computes various evaluation metrics to quantify its performance. Below are the key features and workflow of the evaluator:
	
	\subsubsection{Key Features}
	
	\begin{itemize}
		\item \textbf{Model Loading and Initialization}:
		\begin{itemize}
			\item Supports loading both \texttt{GNNModel} (simple GNN) and \texttt{GeneralizedGNN}.
			\item Automatically restores the model's weights from a specified directory (\texttt{save\_dir}).
			\item Configurable model parameters such as hidden dimensions, number of layers, and variant type (e.g., GCN, GraphSAGE, GAT).
		\end{itemize}
		
		\item \textbf{Dataset Compatibility}:
		\begin{itemize}
			\item Supports evaluation on various datasets (default: \texttt{Cora}).
		\end{itemize}
		
		\item \textbf{Evaluation Metrics}:
		\begin{itemize}
			\item Computes a range of standard metrics, including:
			\begin{enumerate}
				\item \textbf{Accuracy}: Proportion of correctly classified nodes.
				\item \textbf{Precision}: Measure of true positive predictions relative to all positive predictions.
				\item \textbf{Recall}: Measure of true positive predictions relative to all actual positives.
				\item \textbf{F1 Score}: Harmonic mean of precision and recall, representing a balance between the two.
			\end{enumerate}
		\end{itemize}
		
		\item \textbf{Metadata Updates}:
		\begin{itemize}
			\item Updates the saved model's metadata with evaluation results for reproducibility and tracking.
		\end{itemize}
	\end{itemize}
	
	\subsubsection{Workflow}
	
	The evaluation process is as follows:
	\begin{enumerate}
		\item Load the pre-trained model and dataset.
		\item Move the data and model to the appropriate device.
		\item Perform inference on the test nodes (\texttt{test\_mask}) to obtain predicted class labels.
		\item Compute evaluation metrics by comparing predictions against true labels.
		\item Print the metrics and save them as part of the model's metadata.
	\end{enumerate}}
	
	\subsection{Cross-Dataset Evaluator}
	\configuratedText{
	The \texttt{CrossDatasetEvaluator} class provides a comprehensive framework for evaluating the performance of a pre-trained GNN model across multiple datasets. This process assesses the model's generalizability and adaptability to datasets with varying characteristics, such as differing feature dimensions or class distributions.
	
	\subsubsection{Key Features}
	
	\begin{itemize}
		\item \textbf{Support for Multiple Datasets}:
		\begin{itemize}
			\item The class can evaluate the model on a predefined set of datasets (default: \texttt{CiteSeer} and \texttt{PubMed}).
		\end{itemize}
		
		\item \textbf{Feature Dimension Handling}:
		\begin{itemize}
			\item Automatically detects mismatches between the feature dimensions of the dataset and the model's expected input.
			\item Supports multiple techniques for resolving these mismatches:
			\begin{enumerate}
				\item \textbf{Zero Padding}: Adds zero-valued features to match the target dimension.
				\item \textbf{Replication}: Duplicates existing features to fill the required dimensions.
				\item \textbf{Dimensionality Reduction (PCA)}: Reduces features to the target dimension for datasets with higher dimensionalities.
			\end{enumerate}
		\end{itemize}
		
		\item \textbf{Evaluation Metrics}:
		\begin{itemize}
			\item Computes standard metrics such as accuracy, precision, recall, and F1 score for node classification tasks.
			\item Collects and stores dataset-specific information, including the number of nodes, features, and classes.
		\end{itemize}
		
		\item \textbf{Metadata Integration}:
		\begin{itemize}
			\item Stores cross-dataset evaluation results, including metrics and dimension-handling techniques, in the model's metadata for future reference.
		\end{itemize}
	\end{itemize}
	
	\subsubsection{Workflow}
	
	The evaluation process follows these steps:
	\begin{enumerate}
		\item Load the pre-trained model and move it to the appropriate computation device.
		\item Iterate over the specified datasets and preprocess each one.
		\item Detect and handle feature dimension mismatches using the configured technique (\texttt{auto}, \texttt{zero\_pad}, \texttt{replicate}, or \texttt{pca}).
		\item Perform inference on the test set and compute evaluation metrics.
		\item Save the results, including dataset-specific metrics and dimension-handling details, as part of the model's metadata.
	\end{enumerate}}
	
	\subsection{Execution Example}
	\subsubsection{Train, Evaluate, and Cross-Test Using a Single Model Type}
	\configuratedText{
	To train, evaluate, and cross-test a single GNN model, use the following command:
	
	\begin{itemize}
		\item Run the command:
		\begin{tcolorbox}
			\texttt{python main.py single train evaluate cross\_test}
		\end{tcolorbox}
	\end{itemize}
	
	This command derives key parameters from the dataset loader:
	\begin{itemize}
		\item \texttt{input\_dim = self.dataset.num\_features = 1433}:\\ Number of input features from the \texttt{Cora} dataset.
		\item \texttt{hidden\_dim = 256}:\\ Hidden layer dimension (default configuration).
		\item \texttt{num\_classes = 7}:\\ Number of classes, derived from the dataset loader (\texttt{DatasetLoader("Cora")}).
	\end{itemize}
	
	After execution, a folder will be created in the \texttt{saved\_models} directory with a name in the format \texttt{single\_timestamp}. This folder contains:
	\begin{itemize}
		\item \texttt{model.pth}: The trained model's weights.
		\item \texttt{metadata.json}: A file containing detailed metadata about the model, dataset, training configuration, and evaluation metrics.
	\end{itemize}
}
	
	\subsubsection{Perform an Experiment with Different Model Architectures}
	\configuratedText{
	To explore different configurations of the Generalized GNN Model, we can run experiments with varying parameters. This allows us to train and evaluate multiple models and select the one that performs best.
	
	\begin{itemize}
		\item Run the command:
		\begin{tcolorbox}
			\texttt{python main.py experiment}
		\end{tcolorbox}
	\end{itemize}
	
	This command trains, evaluates, and cross-tests models with different configurations, saving the results in the \texttt{saved\_models} directory. The experiments are conducted over a range of parameter values:
	
	\begin{itemize}
		\item \texttt{hidden\_dim\_values}: [64, 128, 256, 512]
		\item \texttt{num\_layers\_values}: [2, 4, 8, 12]
		\item \texttt{variant\_values}: ["gcn", "sage", "gat"]
		\item \texttt{dropout\_values}: [0.3, 0.5, 0.7]
		\item \texttt{use\_residual\_values}: [True, False]
		\item \texttt{use\_layer\_norm\_values}: [True, False]
	\end{itemize}
	
	For each combination of parameters, the experiment performs the following steps:
	\begin{enumerate}
		\item Train the model on the specified dataset.
		\item Evaluate the model's performance on the test set.
		\item Cross-test the model on other datasets (e.g., CiteSeer and PubMed).
	\end{enumerate}
	
	The results, including model weights and evaluation metrics, are saved in the \texttt{saved\_models} directory under subfolders named with the timestamp and corresponding parameter configuration. This allows for systematic comparison and selection of the best-performing model.\\
	
	After running the \texttt{experiment} command, you can execute \texttt{find\_best} command:
	\begin{tcolorbox}
		\texttt{python main.py find\_best}
	\end{tcolorbox}
	It output lists the best-performing models for each dataset along with their respective accuracies. Below is an example output from an experiment:
	
	\begin{itemize}
		\item \textbf{Best Model for PubMed Dataset:}
		\begin{itemize}
			\item \texttt{Model Name: generalized\_20241216\_095352}
			\item \texttt{Accuracy: 0.3900}
		\end{itemize}
		\item \textbf{Best Model for CiteSeer Dataset:}
		\begin{itemize}
			\item \texttt{Model Name: generalized\_20241216\_122132}
			\item \texttt{Accuracy: 0.2680}
		\end{itemize}
		\item \textbf{Best Model for Cora Dataset:}
		\begin{itemize}
			\item \texttt{Model Name: generalized\_20241216\_114950}
			\item \texttt{Accuracy: 0.8220}
		\end{itemize}
	\end{itemize}
	}
	
	\newpage
	\section{Link Prediction}
	\configuratedText{
	For performing link prediction, you can execute \texttt{find\_best} command:
	\begin{tcolorbox}
		\texttt{python main.py link\_prediction}
	\end{tcolorbox}

	\subsection{Link Prediction Implementation}\label{sec:link-prediction}
	
	This section describes the implementation of the link prediction task, which involves training and evaluating a Graph Neural Network (GNN) to predict the existence of edges (links) in a graph. The implementation uses the \texttt{EdgePredictionTrainer} class and the \texttt{GeneralizedGNNForEdgePrediction} model.
	
	\subsubsection{Workflow for Link Prediction}
	The link prediction task is executed using the \texttt{run\_link\_prediction} function, which follows these steps:
	\begin{enumerate}
		\item \textbf{Dataset Loading:} 
		\begin{itemize}
			\item The dataset is loaded using the \texttt{DatasetLoader}, providing node features and edges.
			\item Example datasets include Cora, CiteSeer, and PubMed.
		\end{itemize}
		
		\item \textbf{Model Configuration:} 
		\begin{itemize}
			\item A dictionary of hyperparameters is defined, including:
			\begin{itemize}
				\item Input dimensions (\texttt{input\_dim}) based on the number of node features.
				\item Hidden and output dimensions (\texttt{hidden\_dim} and \texttt{output\_dim}).
				\item Number of layers (\texttt{num\_layers}) and additional settings (e.g., residual connections, dropout).
			\end{itemize}
		\end{itemize}
		
		\item \textbf{Training and Evaluation:} 
		\begin{itemize}
			\item The \texttt{EdgePredictionTrainer} is used to train the GNN for edge prediction.
			\item Actions include:
			\begin{itemize}
				\item \textbf{Training:} Optimizing the model on positive and negative edges.
				\item \textbf{Evaluation:} Calculating metrics such as accuracy, precision, recall, F1-score, and ROC-AUC.
			\end{itemize}
		\end{itemize}
	\end{enumerate}
	
	\subsubsection{EdgePredictionTrainer Class}
	The \texttt{EdgePredictionTrainer} class handles the following:
	
	\begin{itemize}
		\item \textbf{Edge Splitting:} 
		\begin{itemize}
			\item The graph's edges are split into training and validation sets using a random permutation.
			\item Negative edges (non-existent links) are dynamically sampled during training and validation.
		\end{itemize}
		
		\item \textbf{Training Process:} 
		\begin{itemize}
			\item Positive and negative edges are concatenated to create a balanced dataset.
			\item The model predicts scores for edges, and the loss is calculated using binary cross-entropy.
			\item Validation is performed after each epoch, tracking the best model based on the validation ROC-AUC score.
		\end{itemize}
		
		\item \textbf{Evaluation Process:} 
		\begin{itemize}
			\item The model's predictions on the validation/test set are converted to probabilities using the sigmoid function.
			\item Metrics such as accuracy, precision, recall, F1-score, and ROC-AUC are computed to evaluate performance.
		\end{itemize}
	\end{itemize}
	
	\subsubsection{Generalized GNN for Edge Prediction}
	The \texttt{GeneralizedGNNForEdgePrediction} class defines the architecture of the GNN. Key features include:
	\begin{itemize}
		\item Support for different GNN variants (e.g., GCN, GraphSAGE, GAT).
		\item Residual connections and layer normalization for better training stability.
		\item An edge predictor layer to calculate edge scores based on node embeddings.
	\end{itemize}
	
	\subsubsection{Console Output and Metrics}
	Running the link prediction task provides detailed outputs for training and evaluation. Below is an excerpt from the console output:
	
	\begin{tcolorbox}[colback=gray!10, colframe=gray!50, sharp corners, boxrule=0.5mm]
		\texttt{
			Starting link prediction...\\
			Starting training...\\
			Epoch 1/100, Loss: 0.6934, Val AUC: 0.5492\\
			Epoch 2/100, Loss: 0.6929, Val AUC: 0.5969\\
			...\\
			Epoch 100/100, Loss: 0.6181, Val AUC: 0.6160\\
			Starting evaluation...\\
			Evaluation Metrics: \{'accuracy': 0.564, 'precision': 0.650, 'recall': 0.279, 'f1\_score': 0.390, 'roc\_auc': 0.589\}
		}
	\end{tcolorbox}
	
	\textbf{Key Observations:}
	\begin{itemize}
		\item \textbf{Training Metrics:} \\
		The loss decreases during training, and validation ROC-AUC improves, indicating model learning.
		\item \textbf{Evaluation Metrics:} \\
		Metrics like accuracy, precision, recall, F1-score, and ROC-AUC are returned at the end.
	\end{itemize}}

	\newpage
	\subsection{Analysis of Link Prediction Model}\label{sec:link-prediction-analysis}
	\configuratedText{
		\textbf{Model’s Ability to Predict Links:} \\
		The link prediction model effectively identifies existing and non-existing edges by leveraging graph structures and node features. Metrics like ROC-AUC, precision, recall, and F1-score highlight its capability to detect true connections, though trade-offs between precision and recall reflect challenges with sparse or noisy graphs.
		
		\textbf{Practical Applications:} \\
		Link prediction has diverse applications, including:
		\begin{itemize}
			\item \textbf{Recommender Systems:} Suggesting user-item connections or friends in social networks.
			\item \textbf{Knowledge Graph Completion:} Filling in missing relationships in domains like healthcare or search engines.
			\item \textbf{Social Network Analysis:} Discovering communities or improving targeted marketing.
			\item \textbf{Fraud Detection:} Identifying suspicious activities in financial networks.
			\item \textbf{Drug Discovery:} Predicting molecular or protein interactions for new drug targets.
		\end{itemize}
		
		\textbf{Challenges and Considerations:} \\
		Model performance depends on factors such as:
		\begin{itemize}
			\item \textbf{Data Sparsity:} Sparse graphs hinder generalization.
			\item \textbf{Feature Relevance:} Noisy features degrade accuracy.
			\item \textbf{Graph Variability:} Diverse structures may limit transferability.
			\item \textbf{Scalability:} Large graphs require efficient computation.
	\end{itemize}}
	
	

	\newpage
	\section{Node Embeddings Extraction and Visualization}
	
	\subsection{Process of Extracting and Visualizing Node Embeddings}
	\configuratedText{
	The process of extracting and visualizing node embeddings is outlined below:
	
	\textbf{Step 1: Embedding Extraction} \\
	The \texttt{extract\_embeddings} function extracts node embeddings from a specified layer of a trained Generalized GNN (Graph Neural Network). The steps are:
	\begin{itemize}
		\item Metadata is loaded from the model directory to retrieve information about the dataset.
		\item The corresponding dataset is loaded using the \texttt{DatasetLoader}.
		\item A forward hook is registered on the specified layer (default: last layer) to capture its output during forward propagation.
		\item The model processes the graph, and the embeddings are collected.
	\end{itemize}
	
	\textbf{Step 2: Embedding Visualization} \\
	The \texttt{analyze\_and\_visualize\_embeddings} function applies dimensionality reduction techniques (e.g., PCA, t-SNE) to the high-dimensional embeddings and visualizes them in a 2D space. Each point in the plot corresponds to a node, and colors represent different node classes. This step aids in understanding how well the GNN separates nodes of different classes.
	
	\subsection{Visualization of Node Embeddings for the Cora Dataset}
	The visualization generated for the Cora dataset using PCA is shown below:
	
	\sectionCenteredfigure{\figone}{Visualization of Node Embeddings using PCA}{fig-1}
	
	\subsection{Interpretation of Results}
	The visualization provides the following insights:
	\begin{itemize}
		\item \textbf{Cluster Formation:} Nodes of the same class (indicated by color) tend to form clusters. This indicates that the GNN successfully learns class-specific embeddings, separating nodes with similar properties in the embedding space.
		\item \textbf{Inter-class Separation:} Some overlap is visible between certain clusters, suggesting that the model struggles to fully distinguish between similar classes. This could be due to overlapping feature distributions or noisy edges in the graph.
	\end{itemize}
	
	}
	
	\newpage
	\section{Node2Vec Implementation and Results}
	\configuratedText{
	\subsection{Overview}
	Node2Vec is a powerful framework for generating node embeddings from graph data. It employs a biased random walk strategy to explore the graph and captures both local and global structural information through skip-gram modeling. These embeddings are then utilized for downstream tasks such as node classification and link prediction.
	
	\subsection{Execution}
	The Node2Vec process can be executed using the following command:
	\begin{tcolorbox}
		\texttt{python main.py node2vec --dataset Cora --tasks node\_classification link\_prediction}
	\end{tcolorbox}
	
	This command runs Node2Vec on the Cora dataset and evaluates its performance on node classification and link prediction tasks.
	
	\subsection{Training Process}
	During training, Node2Vec optimizes embeddings through random walks and skip-gram modeling. Key steps include:
	\begin{itemize}
		\item Generating random walks of specified length for each node.
		\item Using these walks to create positive and negative examples for skip-gram training.
		\item Optimizing the embeddings via stochastic gradient descent using a logistic loss function.
	\end{itemize}
	
	The training process outputs loss values over multiple epochs, indicating how well the model learns meaningful embeddings. }
	
	\textbf{Example Output:}
	\begin{verbatim}
		Epoch 1/100, Loss: 6.6019
		Epoch 2/100, Loss: 6.4174
		...
		Epoch 100/100, Loss: 1.5908
		Training complete!
	\end{verbatim}
	\configuratedText{
	\subsection{Evaluation Results}
	Once embeddings are generated, Node2Vec evaluates its effectiveness on the following tasks:
	
	\subsubsection{Node Classification}
	Node classification uses the generated embeddings as input features to a logistic regression model. Metrics for classification performance include accuracy, precision, recall, and F1-score. The results for the Cora dataset are as follows:
	\begin{itemize}
		\item \textbf{Accuracy:} 35\%
		\item \textbf{Precision:} 34.48\%
		\item \textbf{Recall:} 35.71\%
		\item \textbf{F1-Score:} 33.94\%
	\end{itemize}
	
	These results highlight the model’s ability to capture class-specific information within the embeddings.
	
	\subsubsection{Link Prediction}
	Link prediction evaluates the embeddings' capability to predict the existence of edges in the graph. Metrics include:
	\begin{itemize}
		\item \textbf{Accuracy:} 75.12\%
		\item \textbf{Precision:} 66.78\%
		\item \textbf{Recall:} 99.98\%
		\item \textbf{F1-Score:} 80.07\%
	\end{itemize}
	
	The high recall indicates that the model is effective in identifying true connections, while the slightly lower precision suggests occasional false positives.
	
	\subsection{Conclusion}
	Node2Vec generates versatile node embeddings that effectively represent graph structures. While performance on node classification tasks highlights room for improvement, link prediction metrics demonstrate its capability to capture meaningful graph relationships. These embeddings are valuable for tasks requiring structural and contextual understanding of graphs.
}
	
	\newpage
	\section{Explainability}
	
	\subsection{Overview}
	Explainability in Graph Neural Networks (GNNs) is crucial for understanding the rationale behind model predictions. By identifying the most important features, nodes, and edges that influence a prediction, we can gain insights into the decision-making process of GNNs. This section outlines the implementation and results of explainability using the GNNExplainer algorithm.
	
	\subsection{Implementation}
	The explainability module uses the GNNExplainer to highlight significant graph components for a given node prediction. The process is executed via the following command:
	\begin{tcolorbox}
		\texttt{python main.py explain --dataset Cora}
	\end{tcolorbox}
	
	Key steps in the implementation include:
	\begin{itemize}
		\item \textbf{Node Explanation}: The GNNExplainer identifies important edges and node features influencing the prediction for a specific node.
		\item \textbf{Edge Masking}: An edge importance mask is generated to indicate which edges were critical for the model's decision.
		\item \textbf{Visualization}: A subgraph centered on the explained node is visualized with highlighted edges and nodes, offering an intuitive understanding of the explanation.
	\end{itemize}
	
	\subsection{Example Results}
	The visualization output for a selected node (Figure~\ref{fig:explainability}) shows:
	\begin{itemize}
		\item Red nodes represent the class to which the node belongs.
		\item Gray nodes indicate other nodes in the subgraph.
		\item Blue edges denote significant edges identified by the explainer as critical to the prediction.
	\end{itemize}
	
	\sectionCenteredfigure{\figtwo}{Visualization of }{fig-2}
	
	\subsection{Analysis}
	The visualization highlights the most critical parts of the graph structure influencing the model’s decision. This information can be used for:
	\begin{itemize}
		\item \textbf{Model Debugging}: Understanding why the model made a specific prediction helps in diagnosing issues such as overfitting or bias.
		\item \textbf{Trust and Interpretability}: Providing insights into the model's decision-making process increases user trust in AI systems.
		\item \textbf{Feature Engineering}: Identifying key features and relationships can guide feature selection or graph preprocessing for improved performance.
	\end{itemize}
	
	
	
\end{document}